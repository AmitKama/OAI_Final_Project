\documentclass[english,12pt]{article}
\usepackage[T1]{fontenc}
\usepackage[latin9]{luainputenc}
\usepackage{babel}
\usepackage{array}
\usepackage{float}
\usepackage{amstext}
\usepackage{amssymb}
\usepackage{graphicx}
\usepackage{upgreek}
\usepackage{cite}
\usepackage{titlesec}
\usepackage{pgfgantt}
\usepackage{pgfgantt}
\usepackage{graphicx}
\usepackage{xcolor}
\usepackage{pdfpages}

\setcounter{tocdepth}{4}
\setcounter{secnumdepth}{4}

\titleformat{\paragraph}
{\normalfont\normalsize\bfseries}{\theparagraph}{1em}{}
\titlespacing*{\paragraph}
{0pt}{3.25ex plus 1ex minus .2ex}{1.5ex plus .2ex}


\usepackage[unicode=true,pdfusetitle,
 bookmarks=true,bookmarksnumbered=false,bookmarksopen=false,
 breaklinks=false,pdfborder={0 0 1},backref=false,colorlinks=false] {hyperref}

% copying some latex magic from Yuval
\usepackage{xspace}
\usepackage{color}
\newif\ifNotes\Notestrue
\newcommand{\swallow}[1]{}
\ifNotes
  \newcommand{\colorcomment}[2]{\leavevmode\unskip\space{\color{#1}[#2]}\xspace}
\else
  \newcommand{\colorcomment}[2]{\leavevmode\unskip\relax}
\fi
\newcommand{\taggedcolorcomment}[3]{\colorcomment{#1}{\textbf{#2}: #3}}
\newcommand{\todo}[1]{\colorcomment{red}{TODO: #1}}
\newcommand{\refs}{\colorcomment{red}{REFS}}
\newcommand{\yossi}[1]{\taggedcolorcomment{green}{Yossi}{#1}}
\newcommand{\amit}[1]{\taggedcolorcomment{blue}{Amit}{#1}}



% source: https://tex.stackexchange.com/a/74782
\newcommand\dblquote[1]{\textquotedblleft #1\textquotedblright}
\newcommand\textbfit[1]{\textbf{\textit{#1}}}


\makeatletter

%%%%%%%%%%%%%%%%%%%%%%%%%%%%%% LyX specific LaTeX commands. % Because html
%converters don't know tabularnewline
\providecommand{\tabularnewline}{\\}
%% A simple dot to overcome graphicx limitations
\newcommand{\lyxdot}{.}


%%%%%%%%%%%%%%%%%%%%%%%%%%%%%% User specified LaTeX commands.
\usepackage{graphicx}

\makeatother

\begin{document}
\begin{titlepage}

\newcommand{\HRule}{\rule{\linewidth}{0.5mm}} 
% Defines a new command for the horizontal lines, change thickness here

\center % Center everything on the page
 

%----------------------------------------------------------------------------------------
%	LOGO SECTION
%----------------------------------------------------------------------------------------

\includegraphics[scale=0.025]{images/bgu.png}\\[1cm] 


 
%----------------------------------------------------------------------------------------

%----------------------------------------------------------------------------------------
%	HEADING SECTIONS
%----------------------------------------------------------------------------------------

% Name of your university/college 
\textsc{\LARGE Ben-Gurion University of the Negev}\\[1.5cm] 

\textsc{\Large Faculty of Engineering Science}\\[0.5cm] 

% Major heading such as course name 
\textsc{\large Department of Software and Information Systems
Engineering}\\[0.5cm] 

% Minor heading such as course title 
\textsc{\large Project in Offensive Artificial Intelligence Course}\\[0.5cm] 

% Minor heading such as course title

%----------------------------------------------------------------------------------------
%	TITLE SECTION
%----------------------------------------------------------------------------------------

\HRule \\[0.4cm]
{ \huge \bfseries OAI Final Project - Robustness of Real Time Deepfakes } \\[0.4cm] 
% Title of your document 
\HRule \\[1.5cm]
 
%----------------------------------------------------------------------------------------
%	AUTHOR SECTION
%----------------------------------------------------------------------------------------

\begin{minipage}{0.4\textwidth}
\begin{flushleft} \large \emph{Author:}\\
Amit Kama % Your name
\end{flushleft}
\end{minipage}
~
\begin{minipage}{0.4\textwidth}
\begin{flushright} \large \emph{Author:} \\
Oren Shvartzman % Your Name
\end{flushright}
\end{minipage}\\[1cm]

% \begin{minipage}{0.4\textwidth}
% \begin{flushleft} \large \emph{Author:}\\
% Barak Yacouel % Your name
% \end{flushleft}
% \end{minipage}
% ~
% \begin{minipage}{0.4\textwidth}
% \begin{flushleft} \large \emph{ }\\
%   % Your name
% \end{flushleft}
% \end{minipage}\\[2cm]
    
% If you don't want a supervisor, uncomment the two lines below and remove the
% section above \Large \emph{Author:}\\
% John \textsc{Smith}\\[3cm] % Your name

%----------------------------------------------------------------------------------------
%	DATE SECTION
%----------------------------------------------------------------------------------------

{\large \today}\\[3cm] 
% Date, change the \today to a set date if you want to be precise


\vfill % Fill the rest of the page with whitespace

\end{titlepage}

\pagebreak{}


\tableofcontents{}

\pagebreak{}


\section{Introduction} \label{introduction}

% Deepfakes and deepfakes in recent years
% What is First Order Model
% It has limitations - drawbacks, that in the literature, there is a growing amount of research dealing with the field.
% Detection (Importance, ways)
% What will be in the report



\pagebreak{}

\section{Methods} \label{methods}

Given the growing need for IDS and IPS for securing ICS, there is a growing need to create
quality datasets for training and evaluating anomaly detection models. This includes setting up
testbeds, collecting data from them, and usually simulating data to provide anomalies. In this section
we present the main datasets developed for the purpose of training and evaluating such models in the
ICS domain.

\pagebreak{}

\section{Experiment and Results} \label{experiment and results}

Below is a comparison table between the abovementioned ICS datasets. The purpose of the table is to be a
decision support tool in selecting a suitable dataset for conducting anomalies detection studies in the
ICS field. Note that only BATADAL dataset does not cointain the class labels for the test set.

\pagebreak{}

\section{Discussion} \label{discussion}

Today, ICS are an integral part of the day-to-day operations of many industries and critical infrastructures,
from power generation, through water treatment, and to oil and gas processing. Cyber attacks against ICS would
lead to disruption to controlling those critical infrastructures and result in harmful physical damage to plants,
environment and humans. According to ICS-CERT, the ICS-targeted attacks are continuously increasing from year to year~\cite{DBLP:conf/dsn/FengLC17}.

\pagebreak{}


\bibliographystyle{plain}
\bibliography{Proposal}

\end{document}