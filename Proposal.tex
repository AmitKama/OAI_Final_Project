\documentclass[english,12pt]{article}
\usepackage[T1]{fontenc}
\usepackage[latin9]{luainputenc}
\usepackage{babel}
\usepackage{array}
\usepackage{float}
\usepackage{amstext}
\usepackage{amssymb}
\usepackage{graphicx}
\usepackage{upgreek}
\usepackage{cite}
\usepackage{titlesec}
\usepackage{pgfgantt}
\usepackage{pgfgantt}
\usepackage{graphicx}
\usepackage{xcolor}
\usepackage{pdfpages}

\setcounter{tocdepth}{4}
\setcounter{secnumdepth}{4}

\titleformat{\paragraph}
{\normalfont\normalsize\bfseries}{\theparagraph}{1em}{}
\titlespacing*{\paragraph}
{0pt}{3.25ex plus 1ex minus .2ex}{1.5ex plus .2ex}


\usepackage[unicode=true,pdfusetitle,
 bookmarks=true,bookmarksnumbered=false,bookmarksopen=false,
 breaklinks=false,pdfborder={0 0 1},backref=false,colorlinks=false] {hyperref}

% copying some latex magic from Yuval
\usepackage{xspace}
\usepackage{color}
\newif\ifNotes\Notestrue
\newcommand{\swallow}[1]{}
\ifNotes
  \newcommand{\colorcomment}[2]{\leavevmode\unskip\space{\color{#1}[#2]}\xspace}
\else
  \newcommand{\colorcomment}[2]{\leavevmode\unskip\relax}
\fi
\newcommand{\taggedcolorcomment}[3]{\colorcomment{#1}{\textbf{#2}: #3}}
\newcommand{\todo}[1]{\colorcomment{red}{TODO: #1}}
\newcommand{\refs}{\colorcomment{red}{REFS}}
\newcommand{\yossi}[1]{\taggedcolorcomment{green}{Yossi}{#1}}
\newcommand{\amit}[1]{\taggedcolorcomment{blue}{Amit}{#1}}



% source: https://tex.stackexchange.com/a/74782
\newcommand\dblquote[1]{\textquotedblleft #1\textquotedblright}
\newcommand\textbfit[1]{\textbf{\textit{#1}}}


\makeatletter

%%%%%%%%%%%%%%%%%%%%%%%%%%%%%% LyX specific LaTeX commands. % Because html
%converters don't know tabularnewline
\providecommand{\tabularnewline}{\\}
%% A simple dot to overcome graphicx limitations
\newcommand{\lyxdot}{.}


%%%%%%%%%%%%%%%%%%%%%%%%%%%%%% User specified LaTeX commands.
\usepackage{graphicx}

\makeatother

\begin{document}
\begin{titlepage}

\newcommand{\HRule}{\rule{\linewidth}{0.5mm}} 
% Defines a new command for the horizontal lines, change thickness here

\center % Center everything on the page
 

%----------------------------------------------------------------------------------------
%	LOGO SECTION
%----------------------------------------------------------------------------------------

\includegraphics[scale=0.025]{images/bgu.png}\\[1cm] 


 
%----------------------------------------------------------------------------------------

%----------------------------------------------------------------------------------------
%	HEADING SECTIONS
%----------------------------------------------------------------------------------------

% Name of your university/college 
\textsc{\LARGE Ben-Gurion University of the Negev}\\[1.5cm] 

\textsc{\Large Faculty of Engineering Science}\\[0.5cm] 

% Major heading such as course name 
\textsc{\large Department of Software and Information Systems
Engineering}\\[0.5cm] 

% Minor heading such as course title 
\textsc{\large Project in Offensive Artificial Intelligence Course}\\[0.5cm] 

% Minor heading such as course title

%----------------------------------------------------------------------------------------
%	TITLE SECTION
%----------------------------------------------------------------------------------------

\HRule \\[0.4cm]
{ \huge \bfseries OAI Final Project - Robustness of Real Time Deepfakes } \\[0.4cm] 
% Title of your document 
\HRule \\[1.5cm]
 
%----------------------------------------------------------------------------------------
%	AUTHOR SECTION
%----------------------------------------------------------------------------------------

\begin{minipage}{0.4\textwidth}
\begin{flushleft} \large \emph{Author:}\\
Amit Kama % Your name
\end{flushleft}
\end{minipage}
~
\begin{minipage}{0.4\textwidth}
\begin{flushright} \large \emph{Author:} \\
Oren Shvartzman % Your Name
\end{flushright}
\end{minipage}\\[1cm]

% \begin{minipage}{0.4\textwidth}
% \begin{flushleft} \large \emph{Author:}\\
% Barak Yacouel % Your name
% \end{flushleft}
% \end{minipage}
% ~
% \begin{minipage}{0.4\textwidth}
% \begin{flushleft} \large \emph{ }\\
%   % Your name
% \end{flushleft}
% \end{minipage}\\[2cm]
    
% If you don't want a supervisor, uncomment the two lines below and remove the
% section above \Large \emph{Author:}\\
% John \textsc{Smith}\\[3cm] % Your name

%----------------------------------------------------------------------------------------
%	DATE SECTION
%----------------------------------------------------------------------------------------

{\large \today}\\[3cm] 
% Date, change the \today to a set date if you want to be precise


\vfill % Fill the rest of the page with whitespace

\end{titlepage}

\pagebreak{}


\tableofcontents{}

\pagebreak{}


\section{Introduction} \label{introduction}

% Deepfakes and deepfakes in recent years
Deepfake is a general term that encompasses the use of deep learning algorithms in order to create
synthetic media, in which one subject in an existing visual and/or audio content is usually replaced
with another's likeness. While fraudulent content has been around for some time, recent advances in
machine vision have posed a major threat to the trust and transparency of the media. Using powerful
machine-learning and deep-learning techniques, deepfakes can now manipulate or generate visual and
audio content that can be more easily misleading.

In recent years, deepfakes have garnered widespread attention for their uses in spreading fake news,
committing financial fraud, creating pornographic materials, and many other disturbing uses. This has
led to a significant need to identify and restrict their use.

% First Order Model
Ever since the introduction of deepfakes, researchers in deep learning have increasingly focused on
this area of research. In particular, they propose methods, as well as practical implementations of
deepfakes is in various fields. Among other methods, in~\cite{DBLP:journals/corr/abs-2003-00196}, Siarohin et al. propose
the first order motion model for image animation. Their framework enables generating a video sequence,
in which an object in a source image is animated according to the motion of a driving video, without
using any annotation or prior information about the specific object to animate. According to them,
once trained on a set of videos depicting objects of the same category, the method can be applied
to any object of it. Based on this method, a number of real time photorealistic avatars have recently
been developed, one of which we will explore in this work.

% It has limitations - drawbacks, that in the literature, there is a growing amount of research dealing with the field.
However, real time avatars are far from perfect, as they are not robust when it comes to an edge cases.
This includes facial gestures in the driving video, objects in the source or target media that make it
difficult to identify facial boundaries, and many more.


% Detection (Importance, ways)
Note that this limitations can be create visual glitches and distortions that can be detected with the
naked eye as well as by software, and therefore can be utilized for deepfakes detection. This observation
drives a growing amount of research dealing with those inaccuracies for dual use -- correcting them in
order to improve the deepfakes' credibility, or exploiting them to distinguish deepfakes from real content.

% What will be in the report
In this work, we evaluate the robustness of real time deepfakes by implementing first order motion
model-based avatarify and examining various edge cases on it. After demonstrating failures in the
implementation, we also provide a method to utilize them for deepfakes detection.


\pagebreak{}

\section{Methods} \label{methods}

Given the growing need for IDS and IPS for securing ICS, there is a growing need to create
quality datasets for training and evaluating anomaly detection models. This includes setting up
testbeds, collecting data from them, and usually simulating data to provide anomalies. In this section
we present the main datasets developed for the purpose of training and evaluating such models in the
ICS domain.

\pagebreak{}

\section{Experiments and Results} \label{experiments and results}

Below is a comparison table between the abovementioned ICS datasets. The purpose of the table is to be a
decision support tool in selecting a suitable dataset for conducting anomalies detection studies in the
ICS field. Note that only BATADAL dataset does not cointain the class labels for the test set.

\pagebreak{}

\section{Discussion} \label{discussion}

Today, ICS are an integral part of the day-to-day operations of many industries and critical infrastructures,
from power generation, through water treatment, and to oil and gas processing. Cyber attacks against ICS would
lead to disruption to controlling those critical infrastructures and result in harmful physical damage to plants,
environment and humans. According to ICS-CERT, the ICS-targeted attacks are continuously increasing from year to year~\cite{DBLP:conf/dsn/FengLC17}.

\pagebreak{}


\bibliographystyle{plain}
\bibliography{Proposal}

\end{document}